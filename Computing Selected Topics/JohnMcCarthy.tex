\documentclass{article}

\usepackage[spanish]{babel}
\usepackage[utf8]{inputenc} % Para usar tildes Unicode
\usepackage[top=0.1cm, bottom=0.5cm, left=2cm, right=2cm]{geometry}
\usepackage[colorlinks=true, urlcolor=blue]{hyperref}

\setlength{\parindent}{1cm}

\title{John McCarthy}
\author{ww ww\~no ww}
\date{} % Para evitar que salga la fecha al llamar a 'maketitle'

\begin{document}
	\pagenumbering{gobble} % Ocultar pagenumber
	\maketitle
	\normalsize{
El investigador John McCarthy fue pionero en el área de la inteligencia artificial que surgió en la década de los 50's, concepto que él mismo formó, así como crear el lenguaje de programación Lisp (lenguaje ampliamente utilizado en el campo de la IA y derivados) así como el concepto de tiempo compartido que se aplica a tareas multiusuario.
\\

Nacido en 1927 y fallecido en 2011, este científico siempre estuvo interesado en las matemáticas de tal manera que estaba convencido que éstas podían hacer frente a las problemáticas de ese momento de una manera diferente, gracias a ello el área de la inteligencia artificial ahora es extensa y diversificada.
\\

En el año 1955 acuñó el término de inteligencia artificial después de concluir sus estudios de posgrado (año 1951) y durante una reunión con otros investigadores de diferentes partes del mundo. Gracias a la gran relevancia que tomó, él mismo decidió crear el lenguaje de programación Lisp pocos años más tarde y que en definitiva se volvería una de las herramientas más importantes en el estudio de esta ciencia.
\\

Mientras el investigador hacía estancias en el MIT, propuso el método de tiempo compartido en sistemas operativos distribuidos que más tarde se volvió un paradigma de computación dominante en el área.
\\

Años más tarde, John McCarthy fundó y dirigió el Laboratorio de Inteligencia Artificial de Stanford donde sus investigaciones se enfocaron a la inteligencia artificial, cómputo interactivo y vehículos inteligentes.
\\

Uno de los logros más destacados del científico y pionero de la inteligencia artificial fue conseguir algunos premios como el Premio Turing (otorgado por la Asociación para la Maquinaria Computacional - ACM) en el año de 1971 precisamente por sus extensas investigaciones en ese campo. Además de ello también consiguió otros premios como el Premio Kyoto de 1988 y la Medalla Nacional de Ciencia en 1990.
\\

Uno de los retos más grandes que McCarthy vio es el hecho de lograr que las computadoras consiguiesen el libre albedrío, es decir la capacidad de una máquina para elegir si desea realizar una tarea o no realizarla.

}

\vspace{1cm}

\section*{Bibliograf\'ia}

\noindent \url{http://www.computerhistory.org/fellowawards/hall/bios/John,McCarthy/}
\\
\noindent \url{http://www-formal.stanford.edu/jmc/}
\\

\large{\hfill \textbf{Hecho en } \LaTeX - \url{ww.com}}

\end{document}