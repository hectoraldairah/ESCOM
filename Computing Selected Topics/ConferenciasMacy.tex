\documentclass{article}

\usepackage[spanish]{babel}
\usepackage[utf8]{inputenc} % Para usar tildes Unicode
\usepackage[top=0.1cm, bottom=0.5cm, left=2cm, right=2cm]{geometry}
\usepackage[colorlinks=true, urlcolor=blue]{hyperref}

\setlength{\parindent}{1cm}

\title{Conferencias Macy}
\author{ww ww\~no ww}
\date{} % Para evitar que salga la fecha al llamar a 'maketitle'

\begin{document}
	\pagenumbering{gobble} % Ocultar pagenumber
	\maketitle
	\normalsize{
La historia de las conferencias Macy se remonta al año 1946 en las que definen como multidisciplinarias ya que se trataron temas relacionados a la cibernética (la ciencia que busca lograr la comunicación entre una entidad viva y una máquina) y teoría de sistemas. Dichas conferencias fueron realizadas durante los años de 1946 a 1953. Sin embargo, todas tenían en común el estudio de la mente humana para así definir una ciencia enfocada a dicha temática, dicho en otras palabras establecer las bases para una ciencia exclusivamente a la mente humana. Cabe destacar que todas se realizaron en el estado de Nueva York en diferentes universidades.
\\

En la primera conferencia su organizador Frank Fremont-Smith fue quien inauguró la misma; en la primera sesión fue presentado algo relacionado al arte en computadoras digitales y neurofisiología, la cual fue impartida por Von Neumman y Lorente de Nó, los cuales se enfocaron en modelos aplicados a las ciencias económicas y políticas.
\\

En la segunda conferencia realizada también en 1946 se retomaron los mismos temas y modelos de la primera conferencia, sin embargo la tercera fue particular ya que uno de los miembros principales, Kurt Lewin falleció inesperadamente. En la cuarta conferencia realizada en el año de 1947 se trataron temas sobre sistemas biológicos y sociales en los que se discutieron mecanismos de arquitecturas neuronales.
\\

Más adelante, en la primavera de 1948 se decidió continuar con el mismo tema de la cuarta conferencia a la que se añadieron teorías sobre el órden y el caos aplicado a sistemas complejos. El siguiente año la siguiente conferencia fue iniciada con una discusión sobre los mecanismos del cerebro humano y su complejidad en su estructura bioquímica; eventualmente se trataron temas sobre traumas infantiles y la psicología.
\\

En el año de 1950 se enfocó mucho sobre la cibernética al igual que las siguientes conferencias; se habló sobre la diferencia entre una mente física y una mente virtual (sus interpretaciones) apoyándose en modelos de algunos investigadores como Pitts y McCulloch. Para el año de 1952 hubo ciertas diferencias debido al cambio de tema ocurrido respecto a las conferencias anteriores ya que la calendarización del mismo no era convincente, incluso más adelante se evaluó la complejidad en sistemas biológicos y las matemáticas.
\\

Finalmente la última conferencia se hizo énfasis en cómo los mecanismos neuronales serían capaces de reconocer figuras y patrones musicales; a diferencia de las otras ésta fue más breve debido a la situación organizacional que tenían los encargados anteriormente.

}

\vspace{1cm}

\section*{Bibliograf\'ia}

\noindent \url{http://www.asc-cybernetics.org/foundations/history/MacySummary.htm}
\\
\noindent \url{http://www.asc-cybernetics.org/foundations/history2.htm}
\\

\large{\hfill \textbf{Hecho en } \LaTeX - \url{ww.com}}

\end{document}