\documentclass{article}

\usepackage[spanish]{babel}
\usepackage[utf8]{inputenc} % Para usar tildes Unicode
\usepackage[top=0.1cm, bottom=0.5cm, left=2cm, right=2cm]{geometry}
\usepackage[colorlinks=true, urlcolor=blue]{hyperref}

\setlength{\parindent}{1cm}

\title{Edward Fredkin}
\author{ww ww\~no ww}
\date{} % Para evitar que salga la fecha al llamar a 'maketitle'

\begin{document}
	\pagenumbering{gobble} % Ocultar pagenumber
	\maketitle
	\normalsize{
Uno de los pioneros de la física digital, ciencia que tiene perspectivas sobre el universo que puede describirse por información y por lo tanto es computable, E. Fredkin es un reconocido investigador de la Universidad Carnegie Mellon que ha realizado contribuciones importantes a la computación reversible y los autómatas celulares.
\\

Una de sus ideas principales es ver a las computadoras digitales como modelos de los procesos básicos de la física, razón por la cual existe el área de la física digital ya que él mismo la impullsó partiendo de esta idea.
\\

A pesar que el científico estuvo interesado en la física, él también ha colaborado de gran forma en el área de la computación, principalmente al escribir un ensamblador para la PDP-1 de la cual previamente sugirió a la institución que la compraran para así impulsar la investigación; además de lo anterior también colaboró en el área de la inteligencia artificial.
\\

Además de eso, en los años siguientes estuvo impartiendo clases en el MIT y Universidad de Boston, así como apoyar y trabajar en proyectos de investigación con sus alumnos.
\\

Aunque de que el investigador se dedicó a las matemáticas, también está fuertemente interesado en la computación ya que ha hecho notables colaboraciones como el modelo de  computadora Billiard-Ball o la compuerta Fredkin; esta última siendo un circuito para cómputo reversible y que tiene la característica de realizar todas las operaciones aritméticas posibles si se interconectan varias de estas compuertas.
\\

Fue tal su colaboración con esta ciencia que la Universidad Carnegie Mellon designó un premio con su nombre, el cual consiste en un pago de \$100 000 USD a aquella persona o grupo de investigación capaz de sea campeón en el Campeonato Mundial de Ajedrez. Evidentemente este concurso establecido junto a la Fundación Fredkin de Cambridge tiene como propósito avanzar en las investigaciones sobre la inteligencia artificial utilizando la compleja ciencia que existe detrás del ajedrez.
}

\vspace{1cm}

\section*{Bibliograf\'ia}

\noindent \url{https://chessprogramming.wikispaces.com/Edward+Fredkin}
\\
\noindent \url{http://www.bottomlayer.com/bottom/finite-all.html}
\\

\large{\hfill \textbf{Hecho en } \LaTeX - \url{ww.com}}

\end{document}