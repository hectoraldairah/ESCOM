\documentclass{article}

\usepackage[spanish]{babel}
\usepackage[utf8]{inputenc} % Para usar tildes Unicode
\usepackage[top=2cm, bottom=2cm, left=2cm, right=2cm]{geometry}
\usepackage[colorlinks=true, urlcolor=blue]{hyperref}

\setlength{\parindent}{1cm}

\title{Desarrollo de la rob\'otica en México}
\author{ww ww\~no ww}
\date{} % Para evitar que salga la fecha al llamar a 'maketitle'

\begin{document}
	\pagenumbering{gobble} % Ocultar pagenumber
	\maketitle
	\normalsize{
El art\'iculo habla sobre una comparativa entre los avances tecnol\'ogicos aplicados a la rob\'otica en países de primer mundo como Inglaterra y Jap\'on así como en pa\'ises en v\'ias de desarrollo tales como M\'exico, donde la calidad educativa es muy diferente y por consecuencia el desarrollo cient\'ifico tambi\'en.
\\

El texto detalla la visita a algunos centros de investigación para conocer un poco sobre algunos proyectos de robots que han desarrollado científicos durante varios años y mediante detalladas investigaciones en campos de la física, química, electrónica, computación y eventualmente las matemáticas.
\\

La visita comienza en el Laboratorio de Robótica de Bristol y en el Laboratorio de Sistemas Autónomos Inteligentes en la ciudad de Bristol, Inglaterra, donde el profesor Chris Melhuish coordina investigaciones en las áreas de robótica aerea, médica, seguridad, humanoide, entre otros; siendo uno de ellos el proyecto o concepto de EcoBot que busca imitar animales y su sistema de navegación utilizando otros dispositivos como antenas o bigotes en lugar de ojos.
\\

También en Bristol, se encuentra el Centro Internacional de Computación No Convencional en el que se busca desarrollar robots de una forma conocida como “computación no convencional”. El profesor Andrew Adamatzky y otros compañeros desarrollaron un robot que se basa en la reacción química Belousov-Zhabotinski (una reacción química oscilante que no está en equilibrio y que tiene mucha relación con la teoría del caos), y de acuerdo al resultado de la reacción se determina el sistema de navegación del robot.
\\

En Japón se encuentra el Miraikan o Museo Nacional de Ciencias Emergentes e Innovación y el Instituto Nacional de Tecnología y Ciencia Industrial Avanzada donde se enfocan específicamente al desarrollo de humanoides. Aquí en este país los autores destacan el sistema educativo en Japón (donde se practica la educación e interés por la ciencia desde la temprana edad) respecto al pésimo sistema educativo mexicano y el poco interés cultural de la sociedad mexicana.
\\

Finalmente se habla un poco sobre el intento de expandir la robótica en México a un nivel educativo y de desarrollo donde solamente destacan algunas instituciones como el CONACyT, IPN, UNAM, INAOE y CINVESTAV. Los problemas más severos en nuestra sociedad para impulsar proyectos de esta magnitud son la poca interacción, la poca continuidad de los mismos y la falta de visión a nivel académico e industrial, conllevando esto a adquirir carísimos proyectos de robótica importados en lugar de apoyar iniciativas mexicanas. Finalmente la robótica en México ha tenido éxito gracias a la minirobótica que ha despertado interés en unos cuantos estudiantes, y que aunque no ha provocado grandes avances científicos debido a su relativa facilidad y bajo costo, ha servido para intentar impulsar un poco a nuestro país en áreas de la tecnología moderna.
}

\vspace{2cm}

\section*{Bibliograf\'ia}

\noindent \url{http://eprints.uwe.ac.uk/18774/}
\\

\large{\hfill \textbf{Hecho en } \LaTeX - \url{ww.com}}

\end{document}