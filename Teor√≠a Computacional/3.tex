\documentclass{article}

\usepackage[spanish]{babel}
\usepackage[utf8]{inputenc} % Para usar tildes Unicode
\usepackage[top=0.1cm, bottom=0.5cm, left=2cm, right=2cm]{geometry}
\usepackage[colorlinks=true, urlcolor=blue]{hyperref}

\setlength{\parindent}{1cm}

\title{¿Es adecuada la fuente de información disponible para \\los estudiantes en computación y áreas afines en México?}
\author{ww ww\~no ww}
\date{} % Para evitar que salga la fecha al llamar a 'maketitle'

\begin{document}
	\pagenumbering{gobble} % Ocultar pagenumber
	\maketitle
	\normalsize{
En el escrito se hace una pequeña discusión sobre los problemas que los alumnos y profesores tienen frecuentemente en las fuentes de información cercanas; además al final del mismo se incluyen algunos enlaces a material digital y a libros recomendados para estudiantes de ciencias de la computación.
\\

Debido a las enormes diferencias sociales que involucran temas tan delicados como religión y economía existentes en todo el país mexicano, implica un resultado negativo en la formación académica de alumnos y profesores que principalmente viven en zonas marginadas o con dificil acceso a la información y los avances científicos-tecnológicos.
\\

Los estudiantes que están ya en su etapa de formación profesional generalmente se encuentran llenos de falsas esperanzas sobre su futuro, generadas principalmente por instituciones y sus ideologías que convencen a la sociedad pero sin embargo no son capaces de cumplir con las expectativas generadas tanto por los medios de comunicación y por la sociedad en general. Análogamente esto se ve como en una típica campaña política, donde se convence erróneamente a mucha gente y posteriormente tiende a aumentar su mediocridad o bajo rendimiento en las cosas que debe realizar.
\\

En la parte de los profesores de dichas instituciones, más allá de criticar la acción-reacción de “las cosas son peor a lo esperado”, deben plantear estrategias de solución a dichos problemas y por supuesto aplicarlas.
\\

Un problema fuerte de la baja motivación de los estudiantes en niveles superiores y el aumento de deserción escolar se debe a la formación académica deficiente hecha en niveles anteriores, lo cual afecta a los alumnos e inclusive a los profesores.
\\

Para mala suerte, existen muy pocos profesores preocupados por la adecuada formación de los estudiantes y que encima dicha actitud se vea denigrada por las políticas existentes en las instituciones, haciendo una imitación a pequeña escala del sistema que nos rige mundialmente donde solamente unos cuantos salen beneficiados sin importarles el bienestar de otros, todo ello generado por administrativos de visión individualista e involucrados en intereses políticos y económicos para beneficio personal.
\\

Pero la culpa de esto también se atribuye a la mentalidad poco agradable de la sociedad (en este caso en particular la sociedad mexicana) la cual es demasiado conformista y que termina por obstruir el proceso de enseñanza-aprendizaje.
\\

Del otro lado, en el ámbito profesional y con personas “preparadas” parece que el ambiente es distinto pero en realidad no cambia mucho ya que los problemas siguen siendo los mismos, aunque ahora muchos investigadores buscan beneficios personales para sentirse más importantes y ser reconocidos; aunque en realidad la sociedad es quien debe dar reconocimiento a los buenos investigadores por sus acciones y no por los puestos ocupados o el sueldo obtenido.
\\

Este es el reflejo de una miserable sociedad como la mexicana, uno de los tantos problemas que sufre y que sin embargo nadie resuelve por dos razones: intereses personales o por severas dificultades puestas por otros para evitar sacarla adelante y proteger de nueva cuenta sus intereses personales; sin duda una lectura y opinión de los autores muy interesante.
}

\vspace{1cm}

\section*{Bibliograf\'ia}

\noindent \url{http://eprints.uwe.ac.uk/10648/}
\\

\large{\hfill \textbf{Hecho en } \LaTeX - \url{ww.com}}

\end{document}