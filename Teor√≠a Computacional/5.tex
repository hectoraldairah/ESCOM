\documentclass{article}

\usepackage[spanish]{babel}
\usepackage[utf8]{inputenc} % Para usar tildes Unicode
\usepackage[top=2cm, bottom=2cm, left=2cm, right=2cm]{geometry}
\usepackage[colorlinks=true, urlcolor=blue]{hyperref}

\setlength{\parindent}{1cm}

\title{La computación en México y la influencia \\de H.V. McIntosh en su desarollo}
\author{ww ww\~no ww}
\date{} % Para evitar que salga la fecha al llamar a 'maketitle'

\begin{document}
	\pagenumbering{gobble} % Ocultar pagenumber
	\maketitle
	\normalsize{
Se define a la computación como aquella disciplina que estudia sistemáticamente procesos algorítmicos los cuales procesan información, y en la que surge la duda de ¿qué puede automatizarse eficientemente?. Existen dos asociaciones internacionales relacionadas a las ciencias de la computación: la ACM (Association for Computing Machinery) y la Computer Society de la IEEE. En cambio, en México no han habido grandes aportes a esta rama de la ciencia, por lo que todo se reduce a logros reconocidos localmente.
\\

Harold V. McIntosh es un reconocido doctor que llegó a México y ha trabajado en instituciones como el CINVESTAV, el Centro de Cálculo Electrónico de la UNAM, en la ESFM como profesor, el CENAC e inclusive el Centro Nuclear de México; en todos aquellos dirigió tesis en las que se involucraban trabajos y estudios sobre compiladores de algunos lenguajes de programación.
\\

Otro de sus reconocimientos se debe a un artículo que escribió en 1971, el cual fue muy citado por Herbert Goldstein, quien fuera autor de un libro de mecánica clásica muy reconocido.
\\

Poco más tarde, en 1973 se fundó la Licenciatura en Computación en la Universidad Autónoma de Puebla, siendo ésta una de las carreras de computación mejor enfocadas a las matemáticas existentes en México.
\\

Hubo varios trabajos sobresalientes posterior al inicio de la carrera tales como sistemas, compiladores e inclusive lenguajes y proyectos de autómatas celulares; todos ellos siendo tesis que están en propiedad de la UAP.
\\

En opinión de algunos, las maestrías no tienen niveles tan altos de dificultad debido a la facilidad de que cualquier persona puede ingresar a ellas y sin importar su profesión, que incluso muchos egresados de la Licenciatura en Computación de la UAP no tienen muchos problemas porque ya habían visto esos temas anteriormente en sus estudios superiores.
\\

La gente talentosa en las ramas de la computación es más dificil mantenerlas gracias a la gran competencia que existe en puestos empresariales debido a los buenos salarios, por lo que muchos investigadores proponen llamar la atención de dichas personas académicamente y a la vez mejorar los planes de estudios a nivel posgrado.
}

\vspace{2cm}

\section*{Bibliograf\'ia}

\noindent \url{http://delta.cs.cinvestav.mx/~mcintosh/cellularautomata/OTHER_TOPICS.html}
\\

\large{\hfill \textbf{Hecho en } \LaTeX - \url{ww.com}}

\end{document}